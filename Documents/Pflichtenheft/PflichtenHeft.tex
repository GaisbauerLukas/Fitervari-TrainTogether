\documentclass[12pt]{article}
\setcounter{secnumdepth}{5}
\usepackage{geometry}                % See geometry.pdf to learn the layout options. There are lots.
\geometry{letterpaper}                   % ... or a4paper or a5paper or ...
\usepackage{graphicx}
\usepackage{amssymb}
\usepackage{amsthm}
\usepackage{epstopdf}
\usepackage[utf8]{inputenc}
\usepackage[usenames,dvipsnames]{color}
\usepackage[table]{xcolor}
\usepackage{hyperref}
\usepackage{german}
\DeclareGraphicsRule{.tif}{png}{.png}{`convert #1 `dirname #1`/`basename #1 .tif`.png}

\theoremstyle{definition}
\newtheorem{example}{Example}

\usepackage{array}
\newcolumntype{L}[1]{>{\raggedright\let\newline\\\arraybackslash\hspace{0pt}}m{#1}}
\newcolumntype{C}[1]{>{\centering\let\newline\\\arraybackslash\hspace{0pt}}m{#1}}
\newcolumntype{R}[1]{>{\raggedleft\let\newline\\\arraybackslash\hspace{0pt}}m{#1}}


\newcommand{\projektname}{Fitervari: Train Together}
\newcommand{\doctype}{Pflichtenheft}
\newcommand{\projektleiter}{F. Gewessler}
\newcommand{\projektmitglieder}{L.Gaisbauer, C. Knoll}
%\newcommand{\documentstatus}{Submitted}
%\newcommand{\documentstatus}{Released}
\newcommand{\version}{V. 1.0}

\begin{document}
\begin{titlepage}
\begin{flushright}
\includegraphics[scale=.5]{htlleondinglogo.png}\\
\end{flushright}

\vspace{10em}

\begin{flushright}

\end{flushright}
\begin{center}
{\LARGE \doctype} \\[3em]
\end{center}
\pagebreak


\begin{flushleft}
\begin{tabular}{|l|l|}
\hline
Projektname & \projektname \\ \hline
Projektleiter & \projektleiter \\ \hline
Projektmitglierder & \projektmitglieder \\ \hline
Version & \version \\ \hline
\end{tabular}
\end{flushleft}

\end{titlepage}

\tableofcontents
\pagebreak

\section{Ist-Zustand}
Oft gehen Menschen heutzutage allein trainieren, da sie keinen haben, der mit ihnen das Workout teilt, oder da ihre normalen Partner keine Zeit haben. Ziel von Fitervari ist es, Menschen zusammen zu bringen, ihr Workout zu teilen und dabei ihren Fortschritt festzuhalten.
Der Markt für Fitness-Tracking Apps ist hart umkämpft und  vollgestopft mit Apps, auch von großen Namen. Allerdings lassen viele der kostenlosen Lösungen zu wünschen übrig und sind mit ihren Features nur eingeschrängt und die wirklich guten Apps verstecken ihre Features meist hinter einer mächtigen Paywall, die oft nicht gerechtfertigt ist. Fitervari soll hier abhilfe schaffen und soll schönes und minimalistisches Tool zum überwachen und festhalten des eigenen Fortschritts sein.


\pagebreak

\section{Zielsetzung}

\subsection{Ziele}
\begin{itemize}
\item Der Benutzer kann unbegrenzt viele Workouts hinzufügen und diese abtrainieren.
\item Der Benutzer kann sein Fortschritt in den jeweiligen Workouts über zahlreiche Diagramme und Übersichten presentiert.
\item Der Benutzer wird informiert, ob ein anderer Nutzer in seinem Fitnessstudio ein Workout teilen möchte und welche Übungen dieses Workout enthält.
\end{itemize}
\subsection{Erweiterungen}
\begin{itemize}
\item Vorschläge für personalisierte Workouts
\end{itemize}

\subsection{Nicht-Ziele}
\begin{itemize}
\item Vorschläge von diversen Ernährungsplänen

\end{itemize}

\subsection{Alleinstellungsmerkmal}
Die Funktion sein Workout mit anderen zu teilen und diese zum Workout einzuladen, ist in der Industrie einzigartig. Desweiteren wird Fitervari: Train Together über eine minimalistische und dennoch mächtige UI verfügen, die alle wichtigen Features zum tracken vom eigens zusammengestellten Workouts beinhaltet.

\pagebreak

\section{Anforderungen}

\subsubsection{App}

Um eine gute Userexperience sicher zu stellen, muss die App sehr simpel strukturiert und einfach zu verwenden sein. Ebenso wichtig ist es, dass die App auf allen Platformen (Android, IOS, Web) verfügbar ist um die Menge an potentiellen Nutzern nicht einzuschränken.

\paragraph{Kommunikationsstruktur}
\begin{center}

\includegraphics[width=15cm]{Kommunikationsdiagramm.jpg}

\end{center}
\pagebreak

\paragraph{Graphische Benutzeroberfläche}
\begin{flushleft}
Unsere Clientapplikation ist in Flutter geschrieben, was die Entwicklung erheblich beschleunigt. Desweiteren legen wir den Fokus auf einfache Bedienbarkeit und elegante, minimalistische Obtik.

\end{flushleft}

\begin{flushleft}

\end{flushleft}

\subsubsection{Server}
Der Server sollte 99\% der Zeit erreichbar sein um eine ununterbrochene Erfahrung zu bieten.\\
\paragraph{Datenmodel der Fitervari Datenbank}
\begin{center}

\includegraphics[width=15cm, height=12.6cm]{datenmodel.png}

\end{center}
\pagebreak

\subsubsection{Usecases}

\includegraphics[width=15cm, height=18.5cm]{UseCase_Diagramm.jpg}

\newpage

\paragraph{App einrichten}
\begin{center}

\end{center}
\begin{center}
\begin{tabular}{| L{3cm} | C{12cm} |}
\hline
App installieren & Die Fitervari: Train Together-App wird sowohl im Google Play Store, als auch im App Store verfügbar sein. Um diese dann im jeweiligen Store zu finden, einfach in den Suchbereich „Fitervari“ eingeben, die App auswählen und diese dann wie gewohnt auf das Mobiltelefon installieren.\\

\hline
Registration & Zur Identifikation wird sich der User registrieren müssen. Der User wird sich also beim Starten der App einloggen müssen, damit die jeweiligen Daten geladen werden können.\\


\hline
\end{tabular}
\end{center}

\pagebreak
\subsubsection{Aktivitätsdiagramme}
\paragraph{Hardware über App verbinden}
\begin{center}

\includegraphics[width=11cm, height=17cm]{Aktivitaetsdiagramm.jpg}

\end{center}

\pagebreak
\section{Entwicklung}

\subsection{Frontend}

Die Frontend-Entwicklung ist mit dem Crossplatform-Framework Flutter, entwickelt von Google, umgesetzt.

\subsection{Backend}

Die Backend-Entwicklung ist in der Sprache Kotlin umgesetzt. Desweiteren wird hier das Quarkus-Framework verwendet.


\pagebreak

\section{Angebot}




\pagebreak

\section{Admistratives}

\subsection{Lizenz}

\subsection{Budgetrahmen}

\end{document}
